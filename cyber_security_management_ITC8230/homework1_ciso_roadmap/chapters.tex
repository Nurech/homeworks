\begin{center}
  \resizebox{\textwidth}{!}{%
    \begin{forest}
      for tree={
        draw,
        rounded corners,
        node options={align=center,},
        text width=4cm,
      },
      where level=0{%
        }{%
        folder,
        grow'=1,
        if level=1{%
          before typesetting nodes={child anchor=north},
          edge path'={(!u.parent anchor) -- ++(0,-15pt) -| (.child anchor)}, % increased from -5pt to -15pt
        }{},
      }
      [\bfseries{CISO}, fill=gray!25, parent
      [\bfseries{Foundations}, for tree={fill=brown!25, child}
      [Education
      [Undergraduate\\(Computer Science; Info Sys BA) \cite{HOOPER2016585, Karanja2017, 11646017}]
      [Postgraduate\\(Cybersecurity MSc) \cite{HOOPER2016585}]
      [Certifications\\(CISSP; CSSLP; CCFP; CISM; CISA;)  \cite{HOOPER2016585, 10658980701746577, cotton}]
      ]
      [Skills
      [Technical\\(Network Security; \cite{11646017, 10658980701746577} Cloud \cite{Karanja2017})]
      [Management\\(Risk Assessment; \cite{11646017} Policy \cite{11646017, Karanja2017})]
      [Soft Skills\\(Communication; \cite{11646017, Karanja2017, HOOPER2016585, 2023dasilvajphd} Leadership \cite{2023dasilvajphd})]
      ]
      ]
      [\bfseries{Governance}, for tree={fill=red!25,child}
      [Regulatory
      [Compliance\\(GDPR \cite{monzelo2019role}; HIPAA \cite{Karanja2017,10658980701746577,HOOPER2016585})]
      [Standards\\(ISO 27001 \cite{HOOPER2016585}; NIST \cite{HOOPER2016585})]
      ]
      [Reporting
      [KPIs\\(Incident Metrics \cite{HOOPER2016585,10658980701746577})]
      [Stakeholders\\(Executives \cite{11646017,10658980701746577,HOOPER2016585,Karanja2017}; Board \cite{monzelo2019role, 11646017,Karanja2017})]
      ]
      ]
      [\bfseries{Operations}, for tree={fill=blue!25, child}
      [Infrastructure
      [Network Security\\(Firewalls \cite{11646017,10658980701746577}; IDS/IPS \cite{11646017,10658980701746577})]
      [Endpoint Security\\(Antivirus \cite{11646017,10658980701746577}; EDR)]
      ]
      [Tools \& Platforms
      [SIEM\\(Splunk; ELK Stack)]
      [Vulnerability Scans\\(Nessus; Qualys)]
      ]
      [Incident Response
      [Plan\\(Development; Testing)]
      [Recovery\\(Business Continuity; DRP)]
      ]
      ]
      [\bfseries{Strategy \& Finance}, for tree={fill=green!25, child}
      [Team Building
      [Internal\\(SOC analysts \cite{HOOPER2016585, 10658980701746577,11646017}; Network Eng \cite{monzelo2019role})]
      [External\\(MSSPs \cite{kark2010market}; Consultants \cite{kark2010market})]
      ]
      [Budget \& Finance
      [CapEx\\(Hardware \cite{11646017,HOOPER2016585, kark2010market}; Licenses)]
      [OpEx\\(Maintenance; Training \cite{10658980701746577})]
      ]
      ]
      ]
    \end{forest}
  }
\end{center}


\textbf{Foundations}: The foundational aspects of the CISO role, comprising key educational and skill-based competencies.
Fundamental to understanding cybersecurity theories and technologies.
A bachelor's degree forms the groundwork for specialized knowledge.
Advanced studies provide a deeper understanding of cybersecurity.
Validate expertise and are often required for senior roles.
A versatile skill set is crucial.
Essential for safeguarding an organization's digital assets.
Vital for strategic decision-making.
Necessary for leading teams and influencing stakeholders.

\textbf{Governance}: Governing policies and compliance to avoid risks.
Ensures compliance with laws and regulations like GDPR and HIPAA.
Compliance frameworks that have legal implications.
Set the guidelines and best practices.
Critical for organizational transparency.
Key metrics to measure security performance.
Regular updates are essential for strategic alignment.

\textbf{Operations}: Day-to-day management of security operations.
Key in defending against cyber threats.
These tools help in intrusion detection and prevention.
Protects individual devices within the network.
Instruments for monitoring and defense.
Platforms for real-time analysis of security alerts.
Tools for identifying vulnerabilities.
Plans and actions for security incidents.
A well-thought-out strategy for incident management.
Plans for resuming normal operations post-incident.

\textbf{Strategy \& Finance}: Deals with long-term planning and resourcing.
Teams amplify a CISO's effectiveness.
In-house staff specialized in different areas.
External expertise for specialized tasks.
Effective resource allocation is key.
Investments in long-term assets.
Day-to-day expenses that need to be managed efficiently.
