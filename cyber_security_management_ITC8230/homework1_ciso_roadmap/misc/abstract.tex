\ac{tpr} have become increasingly popular, particularly
in higher education systems, as they enable users to remotely partake in
events. However, this increased usage also presents potential security risks
specific to \ac{tpr}, such as cyber-physical risks, and exposure of sensitive data among other risks.
Current state of art does not adequately
address these unique concerns, leading to a gap in understanding and
mitigating \ac{tpr} related cybersecurity risks.
This thesis aims to map potential security issues,
offer mitigation strategies for found weaknesses, and bridges the gap by
conducting case
studies and expert interviews.
This research will provide organizations utilizing \ac{tpr} with a
better understanding of security risks and effective solutions to protect
their systems and users.

Telepresence robots (TPRs) are a new emerging segment of service robots.
Such new cyber-physical systems are often accompanied by cybersecurity risks which may not have been considered before.
This paper shows the results of risk perception of healthcare practitioners towards telepresence robots
which were gathered after simulation exercises.
Qualitative analysis of the survey results and presented scenarios shows the ingenuity of the participants when it comes to
generating possible cybersecurity risks.
Perceived cybersecurity risks and participant-generated hypothetical scenarios were analyzed if they are actual risks and how would they
impact the organization if such risks were to be exploited.
Our results contribute to a better understand of cyber-physical threats that may originate from the use of telepresence robots in a
medical environment.
We concluded that a more holistic model based approach is needed to further analyze the possible TPRs cybersecurity risks and the
impact such risks might have when integrating such cyber-physical systems into professional medical work setting, however the found
immediate risks(misuse, eavesdropping, information theft, misconfiguration, identification) are not negligible.

The thesis is written in English language and is [number of pages in main document] pages long, including [number] chapters, [number]
figures and [
number] tables.
