\documentclass{article}
\usepackage[utf8]{inputenc}
\usepackage[sfdefault,ultralight]{FiraSans} %% option 'sfdefault' activates Fira Sans as the default text font
\usepackage[T1]{fontenc}
\renewcommand*\oldstylenums[1]{{\firaoldstyle #1}}
\usepackage{appendix}

\usepackage{graphicx}
\usepackage{amsfonts} 
\usepackage{amssymb}
\usepackage{xcolor}
\usepackage{fdsymbol}
\usepackage{wasysym}
\usepackage{multicol}
\usepackage{amsmath,seqsplit}
\usepackage{algorithm, algorithmic}
\usepackage{tcolorbox}
\usepackage{geometry}
 \geometry{
 a4paper,
 right=20mm,
 left=25mm,
 top=25mm,
 bottom=25mm,
 }
 
\usepackage{seqsplit}
\newcommand\foo[2]{%
    \begin{minipage}{#1}
    \seqsplit{#2}
    \end{minipage}
}

\setlength{\columnsep}{1cm}
\setlength\parindent{0pt}

\definecolor{pink}{RGB}{244,194,194}
\definecolor{green}{RGB}{143,188,143}
\definecolor{red}{RGB}{190,0,50}
\definecolor{ttgrey}{rgb}{0.58,0.59,0.69}
\definecolor{ttblue}{rgb}{0.2,0.17,0.38}
\definecolor{ttred}{rgb}{0.67,0.07,0.32}


\begin{document}

\begin{tcolorbox}[colframe=ttblue, colback=ttblue!10]
\begin{center}
\begin{large}
\textbf{First homework Joosep Parts 221963IVCM}
\end{large}
\end{center}
\end{tcolorbox}

\section*{Tasks}

\paragraph{Task 1.} Assume that the Affine cipher is implemented in $\mathbb{Z}_{97}$, not in $\mathbb{Z}_{26}$. (Imagine that we just
extended alphabet, added a set of special symbols. But the first 26 letters stay the same as in English alphabet.)\\

1. For this modified Affine cipher over \( \mathbb{Z}_{97} \), the encryption and decryption functions remain structurally the same but
adapt to the new modulo.

$$\text{Encryption: } Enc(m) = a\cdot m + b (\mod 97)$$
$$\text{Decryption: } Dec(c) = a^{-1}(c - b) (\mod 97)$$


2. For keys I would need to consider all characters that have a multiplicative inverse modulo 97. Since $b$ can be any character from 0
to 96 (incl.), there are 97 choices(keys) for $b$.\\


3. We read English alphabet and get numerical equivalents for B, U, H are 1, 20, and 7. We do equations to solve for a and b.
To find the encryption and decryption keys, solve the following equations to find a and b. I use Extended Euclidean Algorithm, find a.
And then use value a to find b.
\begin{align*}
  &\text{1. Given equations I use first and second to find a and b:} \\
  &81 = a \cdot 1 + b \mod 97 \\
  &71 = a \cdot 20 + b \mod 97 \\
  &37 = a \cdot 7 + b \mod 97 \\
\end{align*}
\begin{align*}
  &\text{2. Solve for } a \text{ and } b: \\
  &\text{First eliminate } b \text{ by subtracting the first equation from the second:} \\
  &(71 - 81) = a \cdot (20 - 1) \\
  &-10 = 19a \\
  &19a = -10 \mod 97 \\
  &19a = 87 \mod 97 \\
\end{align*}
\begin{align*}
  &\text{Now, find } a^{-1} \text{, the multiplicative inverse of } 19 \mod 97 \\
  &a^{-1} = 19^{-1} \mod 97 \\
  &a^{-1} = 66 \mod 97 \\
  & \\
  &\text{Solve for } a: \\
  &a = 66 \cdot 87 \mod 97 \\
  &a = 17 \mod 97 \\
  & \\
  &\text{Now solve for } b \text{ using the first equation:} \\
  &b = 81 - a \cdot 1 \mod 97 \\
  &b = 81 - 17 \cdot 1 \mod 97 \\
  &b = 64 \mod 97 \\
\end{align*}
\begin{align*}
  &\text{So, } (a, b) = (17, 64) \mod 97 \\
  &\text{3. To decrypt } c' = [59, 62, 90], \\
  &\text{use } m' = a^{-1}(c' - b) \mod 97 \\
  & \\
  &\text{First find } a^{-1} \text{, the multiplicative inverse of } a \mod 97: \\
  &a^{-1} = 17^{-1} \mod 97 \\
  &a^{-1} = 69 \mod 97 \\
  & \\
  &\text{Now decrypt each element of } c': \\
  &m'_1 = 69(59 - 64) \mod 97 = 69(-5) \mod 97 = 90 \mod 97 = 90 \quad (\text{X}) \\
  &m'_2 = 69(62 - 64) \mod 97 = 69(-2) \mod 97 = 48 \mod 97 = 48 \quad (\text{Y}) \\
  &m'_3 = 69(90 - 64) \mod 97 = 69(26) \mod 97 = 88 \mod 97 = 88 \quad (\text{Z}) \\
\end{align*}

\paragraph{Task 2.} Look carefully at the following frequency diagrams.
\begin{enumerate}
    \item Well just like we did in class, by following the patterns and deducting possbilites we could come to a 'certain' conclusion by
    analyzing the English alphabet frequenzy. I took the probabilties table from 2nd week. Classical ciphers pdf.
    \begin{enumerate}
            \item Figure 1: Diagram is identical to Diagram 3 this seems to correspond to a normal distribution, so I would say it
            represents plaintext . Distribution
            of Z, Q and J is >0.2 percent and letters such as E, T and A are most highest frequency.
            \item Figure 2: Diagram 2. This looks like plain text distribution however, in addition to shifting the frequnzies have also
            changed places. So it affine cipher.
      \item Figure 3: Diagram 3 this seems to correspond to a normal distribution, so I would say it represents plaintext. Distribution
            of Z, Q and J is >0.2 percent and letters such as E, T and A are most highest frequency.
      \item Figure 4: Diagram 4 -  Vigenère cipher because the keyword causes letters to repeat in the ciphertext, which makes frequency
            more uniform.
      \item Figure 5: Diagram 5 this is shift cipher locations of
            frequnzies have changed locations but the order is same.
    \end{enumerate}
    \item\begin{minipage}{\linewidth}
          \begin{proof}
            \begin{align*}
              &\text{Let the plaintext alphabet be represented as } \mathcal{A} = \{A, B, \ldots, Z\}. \\
              &\text{Let the ciphertext alphabet be represented as } \mathcal{B} = \{A, B, \ldots, Z\}. \\
              &\text{Let the most frequent letter in the plaintext be 'E'.} \\
              &\text{Let the most frequent letter in the ciphertext be 'T'.} \\
              & \\
              &\text{According to Figure 1 or 3, in the plaintext, the most frequent letter is 'E'.} \\
              &\text{According to Figure 5, in the ciphertext, the most frequent letter is 'T'.} \\
              & \\
              &\text{If we consider 'A' to be 0 and 'Z' to be 25, then:} \\
              &E = 4, \quad T = 19 \\
              & \\
              &\text{The shift from 'E' to 'T' can be represented as:} \\
              &\text{Shift} = T - E \\
              &\text{Shift} = 19 - 4 \\
              &\text{Shift} = 15 \\
              & \\
              &\text{Therefore, the shift cipher has shifted the letters by 15 places.}
            \end{align*}
          \end{proof}
    \end{minipage}
\end{enumerate}


\paragraph{Task 3.} Assuming that the rate of English language is $1.8$, find unicity distance of affine cipher.

\paragraph{Task 4.} Suppose you intercepted the following ciphertext $c = 00010010 \quad 00000111 \quad 11101010$. You know that a 3-letter word was encrypted using one-time pad (to convert letters to binary strings ASCII table was used). Can you bruteforce keys and learn the message that was encrypted?

\paragraph{Task 5.}
You have intercepted the following ciphertext encrypted using Vigenere cipher. You have a cryptoanalyst friend who can help you break the cipher, but he asked you to \textbf{find key length}. \\

\paragraph{Bonus Task} 


\end{document}
